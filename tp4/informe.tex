\documentclass[a4paper]{article}

\usepackage[spanish]{babel} % Le indicamos a LaTeX que vamos a escribir en español.
\usepackage[utf8]{inputenc} % Permite utilizar tildes y eñes normalmente
\usepackage{caratula} % Se puede descargar en ~> https://github.com/bcardiff/dc-tex


\textheight=25cm
\textwidth=18cm
\topmargin=-1.5cm
\oddsidemargin=-1cm 

\begin{document} % Todo lo que escribamos a partir de aca va a aparecer en el documento.

%fran
%\sloppy

% Completar los datos de la caratula
\titulo{TP4 - Planificacion de caminos utilizando RRT} 
\fecha{\today}
\materia{Introducción a la Robótica Móvil}
\grupo{Grupo (número de grupo)}

% Completar los integrantes del grupo:)
\integrante{Schmit, Matias}{714/11}{matias.schmit@gmail.com}
\integrante{Negri, Franco}{893/13}{franconegri2004@hotmail.com}

\maketitle

En los ultimos años se ha visto un gran avance en el campo de la robotica y en tareas automatizables. En particular, los robots omnidireccionales se han empezado a utilizar tanto en la industria, por su capacidad de maniobrar en espacios reducidos, como en la domotica por su comodidad.

En este trabajo practico implementaremos un sistema que permita la realizacion de trayectorias de manera segura y presisa utilizando un robot omnidireccional. Para ello utilizaremos el entorno de desarrollo ROS y el entorno de simulación V-Rep.

El robot en particular sobre el cual trabajaremos contará con cuatro ruedas fijas con diseño Mecanum. De este modo, parte de nuestro trabajo será implementar un modelo cinematico acorde que nos permita tanto controlar la velocidad como estimar la velocidad lineal y angular a partir de la informacion sensada por las cuatro ruedas.


\end{document}
