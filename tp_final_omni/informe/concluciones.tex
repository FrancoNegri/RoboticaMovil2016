%Como conclusiones de este trabajo podemos ver las ventajas de implementar en primer lugar un modelo cinematico para este robot dado y luego sobre el, un modelo de lazo cerrado y finalmente un filtro de kalman, viendo en cada paso las mejoras el la estimación de la pose.

%Ademas, pudimos observar de manera experimental como se ve afectada la presición de la odometría para diferentes restricciones sobre las velocidades del robot.

Como conclusiones finales de este trabajo se logró llevar a la práctica una gran cantidad de los conceptos vistos durante la cursada, a partir del análisis hecho sobre un nuevo modelo se pudo pasar de un sistema resuelto (modelo diferencial) a este que inicialmente resultaba una incógnita. 

A medida que implementamos los distintos componentes necesarios para conseguir el objetivo propuesto adquirimos conocimientos sumamente útiles sobre el sistema bajo estudio, la herramienta de desarrollo y temas anteriormente abordados (cinemática, seguimiento de trayectorias, filtro de kalman), consideramos que el aprendizaje de esto resulta muy enriquecedor 

Centrándonos en los resultados observamos que nuestra implementación presenta buenos resultados y en varias trayectorias no triviales se logró la navegación autónoma sobre dentro del entorno simulado. La experimentación nos permitió ver las ventajas que la correcta implementación de cada una de las partes partícipes en el sistema (Controlador de posición, seguidor de trayectoria, localizador de Kalman, etc) aportan tanto por separado y cuando actúan en conjunto.

%Hay algo mas para poner? Como por ejemplo cosas que no habia que hacer pero se pueden mejorar/agregar al TP? 


