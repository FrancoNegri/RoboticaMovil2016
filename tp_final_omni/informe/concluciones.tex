Como concluciónes de este trabajo podemos ver las ventajas de implementar en primer lugar un modelo cinematico para este robot dado y luego sobre el, un modelo de lazo cerrado y finalmente un filtro de kalman, viendo en cada paso las mejoras el la estimación de la pose.

Ademas, pudimos observar de manera experimental como se ve afectada la presición de la odometría para diferentes restricciones sobre las velocidades del robot.