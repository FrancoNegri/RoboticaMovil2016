En los últimos años se ha visto un gran avance en el campo de la robótica y en tareas automatizables. En particular, los robots omnidireccionales se han empezado a utilizar tanto en la industria, por su capacidad de maniobrar en espacios reducidos, como en la domótica por su comodidad.

En este trabajo practico implementaremos un sistema que permita la realización de trayectorias con alto grado de seguridad y precisión utilizando un robot omnidireccional. Para ello utilizaremos el entorno de desarrollo ROS y el entorno de simulación V-Rep.

El robot en particular sobre el cual trabajaremos contará con cuatro ruedas fijas de tipo Mecanum de 50mm de radio y cuenta con un sensor láser frontal. 

El estudio del movimiento esta dado por el modelado de las fuerzas que afectan el sistema (dinámica) y la matemática presente en la mecánica de movimiento que define las poses del robot en el tiempo (cinemática). En este trabajo nos centramos en lo segundo.

Una vez definido el modelo cinemático se valido su implementación considerando la estimación odometrica del sistema. Se procedió a estudiar el error en la diferencia entre pose obtenida y pose deseada a partir del manejo de la orientación con control a lazo cerrado (feedback). Teniendo este primer modelo de control del error se evaluó la efectividad del seguimiento de una trayectoria sencilla.

Por ultimo se incorporo al sistema un filtro bayesiano, filtro de kalman extendido, para intentar reducir aun mas el error cometido entre pose resultante y esperada. Se muestran experimentos que muestran el impacto del filtro de Kalman sobre la orientación y el seguimiento de la trayectoria previamente usada para el caso anterior.   


%De este modo, parte de nuestro trabajo será implementar un modelo cinematico acorde que nos permita tanto controlar la velocidad como estimar la velocidad lineal y angular a partir de la informacion sensada por las cuatro ruedas.

%No mencionamos que usamos el mismo modulo laser que para el tp del robot diferencial?
