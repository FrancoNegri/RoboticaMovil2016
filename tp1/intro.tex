\section{Introduccion}

En este trabajo utilizaremos diferentes sensores y algoritmos con el objetivo de lograr que un robot móvil terrestre evite y sortee obstáculos de manera efectiva.

Para ello comenzaremos utilizando \textit{bumpers} que permitan detectar cuando el robot choca contra una pared y otro objeto que se encuentre en su camino y reaccionar de manera apropiada para obtener una nueva ruta.

Para que el robot no se vea obligado a chocar contra el objeto para detectarlo, procederemos a utilizar sensores infrarrojos que nos permitiran estimar una distancia entre un obstáculo y el robot. Cuando esa distancia llegue a un valor estipulado buscaremos que reaccione y cambie su recorrido.

Por último plantearemos un algoritmo para utilizar un sonar que nos permita seguir una pared a una distancia determinada.