\section{Sorteo De Obstaculos con Sensores Infrarrojos}
%Sobre el ejercicio 1 y 2 no hay que entregar nada.

%3a:
\subsection{Ejercicio 3}
%Como ya adelantamos en el apartado anterior, como primer instancia de este trabajo utilizaremos bumpers que nos indican si el robot ha colisionado con alg\'un objeto. 

%En caso de que esto haya ocurrido, lo que queremos lograr es que retroceda, gire hacia un costado y continúe su trayectoria hacia adelante.

%Utilizando el siguiente diagrama de estados para guiarnos en el desarrollo del algoritmo final:

Como el enunciado lo indica el robot avanza hacia adelante hasta que uno de los dos sensores que posee detecta un obstáculo en esa dirección debe retroceder hasta estar lo suficientemente alejado (o sea hasta que desaparece de su campo visual a más de 0.25 [metros?]) y luego girar, si el objeto lo detectó a la derecha gira hacia la izquierda y viceversa.

La siguiente máquina de estados refleja este comportamiento:

$$imagen$$

Nos pareció el diseño más claro e intuitivo para reflejar que el comportamiento es distinto cuando se detecta algún objeto con alguno de los dos sensores y el estado al que se procede luego de retroceder también  difiere dependiendo de que lo provocó, por lo tanto debimos usar los dos estados RETROCEDER\_L y RETROCEDER\_R que al alejarse lo suficiente de la pared giran a la derecha e izquierda respectivamente. 



%4a:

\subsection{Ejercicio 4}
Al diseño anterior se le agrega la funcionalidad de un tercer sensor infrarojo ubicado de frente. Partiendo de la misma máquina de estados hay que distinguir el caso en que detectamos un obstaculo con este nuevo sensor (que además mide a otra distancia en comparación con los otros 2). Vale la pena destacar que las acciones que se hacen en la transición entre RETROCEDER\_F y GIRAR\_F son similares a las que hay entre la de RETROCEDER\_L y GIRAR\_L (hab\'ia que girar hacia algun lado cualquiera), entonces es equivalente a tener como transiciones:


AVANZAR  

$[fr < MIN\_DIST\_FRONT \: \lor \: lf < MIN\_DIST\_SIDE] \{ml = mr = -1\} $

$\rightarrow$  RETROCEDER\_LF 

$ [fr > MIN\_DIST\_FRONT \: \lor \: lf > MIN\_DIST\_SIDE] \{ml = 1 \: mr = -1\}$

$\rightarrow$ GIRAR\_L.
%%
%%
$$imagen$$

Si bien se ahorra en la cantidad de estados nos resulta evidente que se pierde claridad y facilidad para interpretar el modelo.



%4a:

\subsection{Ejercicio 5}
Finalmente consideramos un diseño completamente distinto en el cual lo que nuestro robot hace es seguir paralelamente una pared de la que sensa su distancia con un sensor ultrasónico de distancia PING))), la dificultad para este caso está en que al detectar la pared a distinta distancia (nos está diciendo que cambió la pared) debemos girar para seguir manteniendo el movimiento paralelo a la pared. Dependiendo la posición del sonar el giro será a la izquierda o derecha si la distancia con la pared aumentó o disminuyó.

La máquina de estados que refleja este comportamiento es la siguiente:

$$imagen$$

Detectamos como casos borde para analizar cuando se va sensando una pared que interseca perpendicularmente con otra y el caso en que la distancia entre la pared y el robot no es lo suficiente para permitir girar sin chocar con parte de la pared. (Este párrafo no va en el informe, discutir en clase.)