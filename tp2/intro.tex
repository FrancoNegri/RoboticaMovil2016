\section{Introduccion}
En este trabajo experimentaremos con conceptos de cinemática y tecnicas de control, con el proposito de realizar un seguimiento de trayectoria de un robot movil utilizando tecnicas de lazo cerrado.

Como primera instancia buscaremos implementar un algoritmo de control que nos permita converger a una pose objetivo fija. Para ello será necesario operar sobre el sistema de coordenas para restablecerlo de manera adecuada.

Luego experimentaremos con otro tipo de seguimento de trayectoria que imponga una restriccion temporal al robot. Por dar un ejemplo, se le pedirá que luego de 4 segundos de ser iniciado se encuentre en la posición $(2,4)$ con orientación $30º$, a los $7$ segundos que se encuentre en la posición $(5,8)$ con orientación $70º$, etc.

Por ultimo implementaremos y experimentaremos con un algoritmo de seguimiento de persecución. La idea de los mismos es que cuando el robot converja a una pose intermedia deseada, se la cambiaremos por otra mas distante, con el objetivo de suavizar la trayectoria.